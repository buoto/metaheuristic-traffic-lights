\documentclass[a4paper]{article}

\usepackage[english,polish]{babel}
\usepackage{polski}
\usepackage{microtype}
\usepackage[utf8]{inputenc}
\usepackage{gensymb}
\frenchspacing
\usepackage{ragged2e}
\setlength{\RaggedRightParindent}{\parindent}
\usepackage[]{geometry}
\usepackage{indentfirst}
\usepackage{blindtext}
\usepackage{listings}
\usepackage{mathtools}
\usepackage{graphicx}
\usepackage{hyperref}
\usepackage{changepage}
\usepackage{xcolor}
\newcommand{\code}[1]{\texttt{#1}}
%\setcounter{secnumdepth}{1}

\lstset{language=C++,
keywordstyle=\color{blue},
basicstyle=\scriptsize\ttfamily,
commentstyle=\ttfamily\itshape\color{gray},
stringstyle=\ttfamily,
morecomment=[l][\color{magenta}]{\#},
showstringspaces=false,
breaklines=true,
xleftmargin=20pt
}

\title{
Sygnalizacja świetlna
\\ Projekt wstępny
}
\date{}
\author{Michał Błotniak, Adrianna Małkiewicz}

\begin{document}
\RaggedRight

\maketitle

\section{Opis zagadnienia}
Celem projektu będzie rozwiazanie problemu optymalizacji działania sygnalizacji
świetlnej na skrzyżowaniu. Dla podanych funkcji natężenia ruchu w ciągu dnia
poszukiwane będą momenty zmiany świateł, które spowodują minimalizację sumarycznego
czasu spędzonego na skrzyżowaniu.

Skrzyżowanie składa się z 4 dróg wjeżdżających posiadajacych niezależne profile
natężenia, które opisują ile samochodów pojawia się w danej chwili. Samochód stojący
bezpośrednio przed światłami może wjechać na skrzyżowanie jeśli ma zielone światło
oraz na skrzyżowaniu nie przebywa obecnie samochód jadący w tym samym kierunku.
Po wjeździe na skrzyżowanie samochód przebywa na nim pewien czas, ustalony dla
każdej instancji problemu.
Samochody wjeżdżają kolejno na skrzyżowanie do momentu wyczerpania samochodów
w kolejce lub zmiany światła.
(CZY PRZEJAZD PRZEZ SKRZYŻOWANIE ZAJMUJE CZAS?)
(CZY MOŻEMY ZAŁOŻYĆ ŻE SAMOCHODY JADĄ TYLKO PROSTO [tzn. zawsze opuszczają skrzyżowanie
(PRZYJMUJEMY CZAS ZMIANY ŚWIATEŁ?)
w stałym czasie]?)
(CZY FUNKCJE NATĘŻENIA SĄ DYSKRETNE?)
(CZY MOŻEMY PRZYJĄĆ KWANT CZASU?)
\subsection{}
\section{Przestrzeń przeszukiwania}
(CZY PUNKTY TO KROTKI ZAWIERAJĄCE CHWILE ZMIAN ŚWIATEŁ?)
(CZY SĄSIEDZTWO TO DOSTAWIENIE STANU W DOWOLNYM MOMENCIE? NIEOGRANICZONE OKNO?)
\section{Propozycja postaci funkcji celu}
(SUMARYCZNY CZAS SPĘDZONY NA SKRZYŻOWANIU?)
\section{Metoda optymalizacji}
\section{Przewidywane wyniki pracy}
\end{document}
