\documentclass[a4paper]{article}

\usepackage[polish]{babel}
\usepackage{polski}
\usepackage[utf8]{inputenc}
\usepackage[]{geometry}
\usepackage{indentfirst}
\usepackage{ragged2e}

\title {
  Sygnalizacja świetlna
  \\ Projekt wstępny
}

\date{02.05.2017}

\author{Michał Błotniak, Adrianna Małkiewicz}

\begin{document}

\maketitle

\justify

\section{Opis zagadnienia}

Celem projektu będzie rozwiazanie problemu optymalizacji działania sygnalizacji świetlnej na skrzyżowaniu. Dla podanych
funkcji natężenia ruchu w ciągu dnia poszukiwane będą momenty zmiany świateł, które spowodują minimalizację
sumarycznego czasu spędzonego na skrzyżowaniu.

Skrzyżowanie składa się z 4 dróg wjeżdżających posiadajacych niezależne profile natężenia, które opisują ile samochodów
pojawia się w danej chwili. Samochód stojący bezpośrednio przed światłami może wjechać na skrzyżowanie jeśli ma zielone
światło oraz na skrzyżowaniu nie przebywa obecnie samochód jadący w tym samym kierunku. Po wjeździe na skrzyżowanie
samochód przebywa na nim pewien czas, ustalony dla każdej instancji problemu. Samochody wjeżdżają kolejno na
skrzyżowanie do momentu wyczerpania samochodów w kolejce lub zmiany światła.

(CZY PRZEJAZD PRZEZ SKRZYŻOWANIE ZAJMUJE CZAS?)

(CZY MOŻEMY ZAŁOŻYĆ ŻE SAMOCHODY JADĄ TYLKO PROSTO [tzn. zawsze opuszczają skrzyżowanie w stałym czasie]?)

(PRZYJMUJEMY CZAS ZMIANY ŚWIATEŁ?)

(CZY FUNKCJE NATĘŻENIA SĄ DYSKRETNE?)

(CZY MOŻEMY PRZYJĄĆ KWANT CZASU?)

\section{Przestrzeń przeszukiwania}
(CZY PUNKTY TO KROTKI ZAWIERAJĄCE CHWILE ZMIAN ŚWIATEŁ?)

(CZY SĄSIEDZTWO TO DOSTAWIENIE STANU W DOWOLNYM MOMENCIE? NIEOGRANICZONE OKNO?)

\section{Propozycja postaci funkcji celu}
(SUMARYCZNY CZAS SPĘDZONY NA SKRZYŻOWANIU?)

\section{Metoda optymalizacji}

\section{Przewidywane wyniki pracy}

\end{document}
