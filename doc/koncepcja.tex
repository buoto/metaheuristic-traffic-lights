\documentclass[a4paper]{article}

\usepackage[polish]{babel}
\usepackage{polski}
\usepackage[utf8]{inputenc}
\usepackage[]{geometry}
\usepackage{indentfirst}
\usepackage{ragged2e}
\usepackage{hyperref}
\usepackage{mathtools}

\frenchspacing

\title {
  Sygnalizacja świetlna
  \\ Projekt wstępny
}

\date{02.05.2017}

\author{Michał Błotniak, Adrianna Małkiewicz}

\begin{document}

\maketitle

\justify

\section{Opis zagadnienia}

Celem projektu będzie rozwiazanie problemu optymalizacji działania sygnalizacji
świetlnej na skrzyżowaniu. Dla podanych rozkładów losowych natężenia ruchu
w ciągu dnia poszukiwany będzie sposób zmiany świateł, który spowoduje
minimalizację sumarycznego czasu spędzonego na skrzyżowaniu.

Symulowany będzie ruch samochodów w ciągu doby. Dla uproszczenia problemu
zakładamy podział doby na n przedziałów. Skrzyżowanie składa się z 4 dróg, które
nazwane zostały nazwami kierunków geograficznych (NESW). W każdym przedziale
czasowym, z każdego kierunku, przyjeżdza losowa liczba samochodów zgodnie
z określonym dla tego kierunku rozkładem prawdopodobieństwa. W zależności od
pór dnia, opisanych w tablicy~\ref{tab:hours}, wykorzystywane będą różne
rozkłady Poissona z parametrami ustalonymi dla danej instancji problemu.
W każdym z przedziałów poszukiwany będzie stosunek czasu światła zielonego
w kierunku północ-południe (NS) do czasu przedziału.

\begin{table}[ht]
    \centering
    \begin{tabular}{| c | c | c | c |}
        \hline
        Rano & W południe & Wieczorem & W nocy \\ \hline
        6 - 10 & 10 - 15 & 15 - 19 & 19 - 6 \\ \hline
    \end{tabular}
    \caption{Pory dnia z przedziałami czasowymi\label{tab:hours}}
\end{table}

Samochód stojący bezpośrednio przed światłami może wjechać na skrzyżowanie
jeśli ma zielone światło oraz na skrzyżowaniu nie przebywa obecnie samochód
jadący z tego samego kierunku lub z kierunku prostopadłego. Po wjeździe na
skrzyżowanie samochód przebywa na nim losowy czas, zgodny z rozkładem normalnym
o parametrach ustalonych dla danej instancji problemu. Samochody wjeżdżają
kolejno na skrzyżowanie do momentu wyczerpania samochodów w kolejce lub zmiany
światła.

\section{Przestrzeń przeszukiwania}
Przestrzenią przeszukiwań będzie przestrzeń $[0, 1]^n$, gdzie $n$ to liczba
przedziałów czasowych, a każda z wartości od 0 do 1 oznacza stosunek czasu światła
zielonego w kierunku NS do czasu przedziału.

Jako że rozwiązania są punktami w n-wymiarowej przestrzeni metryką będzie
metryka euklidesowa (odległość między dwoma punktami).
% WZÓR NA METRYKE

%Punkty sąsiadują ze sobą, jeśli ich odległość jest mniejsza od promienia
%sąsiedztwa $r$.
Wszystkie wartości czasowe liczone będą w minutach i ich ułamkach.

\section{Propozycja postaci funkcji celu}
Funkcja celu będzie sumarycznym czasem spędzonym na skrzyżowaniu przez wszystkie
samochody w przeciągu całej doby. Czas ten będzie liczony poprzez odtworzenie
symulacji dla danego punktu z przestrzeni problemu. Wartość funkcji będzie
minimalizowana.

\paragraph{Przykład} Niech liczba przedziałów czasowych $n=3$, czas
przejeżdżania samochodów $p=2h$, punkt przestrzeni stanów $x=(0.375, 0.625, 0.5)$
oraz pojawianie się samochodów będzie określone tablicą~\ref{tab:example}.

\begin{table}[ht]
    \centering
    \begin{tabular}{| c | c | c | c | c |}
        \hline
         & N & E & S & W \\ \hline
         $i=0, t=0$ & 2 & 1 & 1 & 2\\ \hline
         $i=1, t=8$ & 1 & 2 & 1 & 2\\ \hline
         $i=2, t=16$ & 1 & 2 & 3 & 1\\ \hline
    \end{tabular}
    \caption{Liczba samochodów przyjeżdżająca z danego kierunku w danej chwili\label{tab:example}}
\end{table}

Wartość funkcji celu będzie liczona ze wzoru:
$$f(x) = \sum\limits_{T} c_i * d_i $$

gdzie T jest zbiorem przedziałów czasu o długości $d_i$, w których czeka pewne $c_i$
samochodów.

\begin{table}[ht]
    \centering
    \resizebox{\textwidth}{!}{
    \begin{tabular}{| c | c | c | c | c | c | c | c | c | c | c | c | c |}
        \hline
         $T_i$     & $[0, 2]$ & $[2, 4]$ & $[4, 6]$ & $[6, 8]$ & $[8, 13]$ & $[13, 15]$ & $[15, 16]$ & $[16, 17]$ & $[17, 19]$ & $[19, 21]$ & $[21, 23]$ & $[23, 24]$ \\ \hline
         $d_i [h]$ & 2 & 2 & 2 & 2 & 5 & 2 & 1 & 1 & 2 & 2 & 2 & 1 \\ \hline
         $c_i$     & 4 & 2 & 1 & 0 & 4 & 2 & 0 & 7 & 5 & 4 & 2 & 1 \\ \hline
    \end{tabular}
}
    \caption{Zbiór przedziałów czasowych $T$ wraz z ich długością $d_i$ oraz
    liczbą czekających samochodów $c_i$, dla punktu $x=(0.375, 0.625, 0.5)$\label{tab:intervals}.}
\end{table}

W symulacji zostają wyliczone wartości dla punktu $x$ zapisane w tablicy~\ref{tab:intervals}. 
Wtedy wartość funkcji celu w punkcie $x$ będzie wynosić:
$$f(x) = \sum\limits_{T} d_i * c_i = 4*2h + 3*2h + 1*2h + 0 + 4*5h + 2*2h + 0 + 7*1h + 5*2h + 4*2h + 2*2h + 1*1h = 70h $$

\section{Metoda optymalizacji}
Do optymalizacji zostanie wykorzystany mutacyjny algorytm ewolucyjny
z krzyżowaniem. Osobniki zostaną sparowane każdy z każdym. Krzyżowanie par będzie
krzyżowaniem uśredniającym z wylosowaną wagą udziału osobnika pierwszego
(z przedziału $[0, 1]$).

Następnie każdy nowy osobnik zostanie poddany mutacji,
czyli przysunięty o losowy wektor ${\rm I\!R}^n$. Współrzędne wektora będą
losowane zgodnie ze standardowym rozkładem normalnym o odchyleniu standardowym
podanym jako parametr algorytmu. Dodatkowo wszystkie współrzędne otrzymanych
osobników będą normalizowane do przedziału $[0, 1]$ przy użyciu operacji modulo 1.

Selekcja będzie selekcją proporcjonalną. Każdemu punktowi zostanie przydzielone
prawdopodobieństwo odwrotnie proporcjonalne do jego rangi. Następnie punkty
zostaną wylosowane przy pomocy metody odwrotności dystrybuanty. W ten sposób
wybrane zostanie nowe pokolenie składające się z $\mu$ punktów.

Pierwsze pokolenie zostanie wygenerowane poprzez wylosowanie punktów z
przestrzeni $[0, 1]^n$.

Algorytm działa tak długo, aż zostanie wygenerowane pokolenie o numerze $T$,
z którego wybierany jest najlepszy osobnik jako wynik działania.

\section{Przewidywane wyniki pracy}
Oczekiwanym rezultatem działania algorytmu jest uzyskanie mniejszego czasu
spędzonego na skrzyżowaniu niż w przypadku stałego cyklu zmiany świateł,
ustalonego a priori.

Przykłady testowe będą odzwierciedlać rzeczywiste sytuacje, gdzie zazwyczaj
ruch jest wzmożony rano i wieczorem, przeciętny w południe oraz niski w nocy.
Różne instancje problemu możemy uzyskać poprzez używanie różnego ziarna
w generatorach losowych.

Wyniki działania algorytmu zostaną przedstawione na wykresie wartości funkcji
celu w zależności od numeru pokolenia. Pokazane będą przebiegi
algorytmu dla pewnego problemu, dodatkowo zestawione z efektem działania
ustalonego z góry cyklu świateł.

\end{document}
