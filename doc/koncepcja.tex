\documentclass[a4paper]{article}

\usepackage[polish]{babel}
\usepackage{polski}
\usepackage[utf8]{inputenc}
\usepackage[]{geometry}
\usepackage{indentfirst}
\usepackage{ragged2e}
\usepackage{hyperref}

\frenchspacing

\title {
  Sygnalizacja świetlna
  \\ Projekt wstępny
}

\date{02.05.2017}

\author{Michał Błotniak, Adrianna Małkiewicz}

\begin{document}

\maketitle

\justify

\section{Opis zagadnienia}

Celem projektu będzie rozwiazanie problemu optymalizacji działania sygnalizacji
świetlnej na skrzyżowaniu. Dla podanych rozkładów losowych natężenia ruchu
w ciągu dnia poszukiwany będzie sposób zmiany świateł, który spowoduje
minimalizację sumarycznego czasu spędzonego na skrzyżowaniu.

Symulowany będzie ruch samochodów w ciągu doby. Dla uproszczenia problemu
zakładamy podział doby na n przedziałów. Skrzyżowanie składa się z 4 dróg, które
nazwane zostały nazwami kierunków geograficznych (NESW). W każdym przedziale
czasowym, z każdego kierunku, przyjeżdza losowa liczba samochodów zgodnie
z określonym dla tego kierunku rozkładem prawdopodobieństwa. W zależności od
pór dnia, opisanych w tablicy~\ref{tab:hours}, wykorzystywane będą różne
rozkłady Poissona z parametrami ustalonymi dla danej instancji problemu.
W każdym z przedziałów poszukiwany będzie stosunek czasu światła zielonego
w kierunku północ-południe (NS) do czasu przedziału.

\begin{table}[ht]
    \centering
    \begin{tabular}{| c | c | c | c |}
        \hline
        Rano & W południe & Wieczorem & W nocy \\ \hline
        6 - 10 & 10 - 15 & 15 - 19 & 19 - 6 \\ \hline
    \end{tabular}
    \caption{Pory dnia z przedziałami czasowymi\label{tab:hours}}
\end{table}

Samochód stojący bezpośrednio przed światłami może wjechać na skrzyżowanie
jeśli ma zielone światło oraz na skrzyżowaniu nie przebywa obecnie samochód
jadący z tego samego kierunku lub z kierunku prostopadłego. Po wjeździe na
skrzyżowanie samochód przebywa na nim losowy czas, zgodny z rozkładem normalnym
o parametrach ustalonych dla danej instancji problemu. Samochody wjeżdżają
kolejno na skrzyżowanie do momentu wyczerpania samochodów w kolejce lub zmiany
światła.

\section{Przestrzeń przeszukiwania}


(CZY PUNKTY TO KROTKI ZAWIERAJĄCE CHWILE ZMIAN ŚWIATEŁ?)

(CZY SĄSIEDZTWO TO DOSTAWIENIE STANU W DOWOLNYM MOMENCIE? NIEOGRANICZONE OKNO?)

\section{Propozycja postaci funkcji celu}
(SUMARYCZNY CZAS SPĘDZONY NA SKRZYŻOWANIU?)

\section{Metoda optymalizacji}

\section{Przewidywane wyniki pracy}

\end{document}
